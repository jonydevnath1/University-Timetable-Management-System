\documentclass[12pt,a4paper]{article}

\usepackage[utf8]{inputenc}
\usepackage{geometry}
\geometry{a4paper, margin=1in}
\usepackage{enumitem}
\usepackage{graphicx}
\usepackage{float}

\begin{document}

% -------------------------
% Title Page
% -------------------------
\begin{titlepage}
    \centering
    \vspace*{\fill} % vertically center
    {\Huge University Timetable Management System (UTMS) \par}
    \vspace{1cm}
    {\Large Software Requirement Specification (SRS) \par}
    \vspace*{\fill} % fill remaining space
    \thispagestyle{empty} % removes page number
\end{titlepage}

\clearpage  % Start main content on a new page

% -------------------------
% Problem Definition Section
% -------------------------
\section{Problem Definition}

In universities, preparing academic timetables is a complex and time-consuming task. Universities have multiple departments, many courses, faculty members, classrooms, and laboratories. In many cases, timetables are created manually using paper or spreadsheet software.

Manual timetable preparation causes several problems.

Major problems include:
\begin{itemize}
    \item A faculty member may be assigned to more than one class at the same time.
    \item Classrooms or laboratories may be allocated to multiple courses at the same time.
    \item Faculty workload may not be distributed evenly.
    \item Updating or modifying timetables takes a lot of time and effort.
    \item Errors are difficult to detect and correct manually.
\end{itemize}

These issues affect academic efficiency and create confusion for faculty members and students.

The University Timetable Management System (UTMS) is a software system designed to automate the creation, management, and modification of university academic timetables. The system reduces manual effort, avoids scheduling conflicts, and improves accuracy.

Objectives of the system:
\begin{itemize}
    \item To automatically generate university timetables.
    \item To prevent faculty and room scheduling conflicts.
    \item To reduce manual work and human errors.
    \item To provide easy access to timetables for university faculty and students.
    \item To find available rooms for extra or urgent classes when needed.
\end{itemize}

\section{Project Scope}

The University Timetable Management System is designed only for universities.

The system:
\begin{itemize}
    \item Manages university departments, courses, subjects, faculty, classrooms, laboratories, and time slots.
    \item Supports different academic activities such as lectures, laboratory sessions, and tutorials.
    \item Generates and maintains department-wise and course-wise timetables.
    \item Allows authorized users to update and modify timetables when required.
    \item Can display available rooms for any given time period.
    \item Supports finding rooms for extra or urgent classes.
    \item Can be extended later as an API-based system for integration with other university systems.
\end{itemize}

\section{Software Requirement Analysis}

\subsection{System Modules}

\textbf{(a) Admin Module}

The Admin module is used by the university academic administration.
\begin{itemize}
    \item Add and manage departments and courses.
    \item Add faculty and subject information.
    \item Define classrooms, laboratories, and available time slots.
    \item Generate university timetables automatically.
    \item Modify and update timetables.
    \item Resolve timetable conflicts.
    \item Check which rooms are free at a given time.
    \item Assign free rooms to regular or extra classes quickly.
\end{itemize}

\textbf{(b) Faculty Module}

The Faculty module is used by university teaching staff.
\begin{itemize}
    \item View assigned teaching timetable.
    \item View subject details and workload information.
    \item Submit timetable change requests (optional).
\end{itemize}

\textbf{(c) Student Module}

The Student module is used by university students.
\begin{itemize}
    \item View class timetable.
    \item View subject-wise and day-wise schedule.
\end{itemize}

\subsection{Functional Requirements}

Functional requirements describe what the system must do.
\begin{itemize}
    \item The system shall allow the admin to create and manage university departments.
    \item The system shall allow the admin to add faculty members, courses, and subjects.
    \item The system shall generate timetables without faculty or room conflicts.
    \item The system shall allow faculty members to view their assigned timetable.
    \item The system shall allow students to view their class timetable.
    \item The system shall allow the admin to update and modify timetables when needed.
    \item The system shall store all timetable and user data in a database.
    \item The system shall support searching timetables by department, faculty, or subject.
    \item The system shall display available rooms for a given time period, showing which rooms are free for scheduling or immediate use.
    \item The system shall allow filtering of available rooms by building, room type, or capacity.
    \item The system shall allow the admin to assign free rooms to regular classes or extra/urgent classes.
\end{itemize}

\subsection{Non-Functional Requirements}

Non-functional requirements describe how the system performs.

\textbf{Performance:}
\begin{itemize}
    \item The system shall generate and display timetables within 2 seconds.
    \item The system shall quickly display available rooms for any time period, including extra/urgent classes.
\end{itemize}

\textbf{Security:}
\begin{itemize}
    \item Only authorized admin users shall be allowed to add or modify data.
    \item Faculty and students shall have read-only access to timetables.
\end{itemize}

\textbf{Reliability:}
\begin{itemize}
    \item The system shall be available during university working hours.
\end{itemize}

\textbf{Usability:}
\begin{itemize}
    \item The system shall provide a simple and user-friendly interface.
    \item Navigation, menus, and buttons shall be clear and easy to understand.
\end{itemize}
